% Appendix Template

\chapter{Ambientación narrativa utilizada en los ensayos}
\label{AppendixB} % For referencing this appendix elsewhere, use \ref{AppendixA}
The World of Aanrah
Aanrah is heavily defined by the magic powers that made it and that ultimately fuel the world. While there is some ammount of technology and 'science', many aspects of real life on Earth are instead tied to magic.

Magic \& the Divine
On Aanrah, divinity is the major drive of all levels of life in some shape or form. All things on Aanrah, animate and inanimate, have a aspect of divinity in them that shapes and defines who and/or what they are. This aspect of divinity serves in place of genetics, and is the base source of much of Aanrah's magic.
Magic itself comes in two forms on Aanrah - inherent and learned. Up until the start of the Third Break, magic was largely restricted to certain races in the form of inherent magic. Inherent magic are magic capabilities that are tied to certain race's divine nature, and are usually a certain sect of magic these races are capable of using naturally. Races like dragons and other divineborn or interloper species are capable of inherent magic, making them significantly more dangrous then those races who did not have any inherent magical capabilities. Races without inherent magic could only learn magic as spells in the way a wizard might - which was, for most of Aanrah's history, notoriously hard to do. Emrul, the Greater God of Magic, sought to keep that kind of information contained and restricted. Inherent magic tends to be more limited then learned magic, and Emrul worried that allowing learned magic to become widely spread could quickly become a problem - and he was right. Since the Third Break, learned magic has spread beyond Emrul's control, and as a result mortals without inherent magic have grown much stronger and capable of pushing back against the races that previously posed serious threats to them - including even the gods.

Technology
On Aanrah, technology exists, but is much more limited. Magic is the driving force behind any sort of machine or automaton, but the science of more object-based technology has not come super far. In some parts of Aanrah, especially those places inhabited by dwarves, one might be able to find early light bulbs or steam trains and trollys - but all of these are still powered predominately by magic.

Aanrah \& Its Neighbors
Aanrah itself is a planet about the size of Earth, revolving around a sun. It is broken down into five continents, which without techtonic plate activity, have remained in the same place they started when the world was made. Initially, Aanrah was orbited by three moons - Honuon, the first and biggest moon; Baza, the second and smallest moon; and Benshuen, the third and middle-sized moon. These three moons together gave early Aanrah a fairly extreme tide, and vastly limited the surface land terrestrial races could live. Over Aanrah's history though, Honuon and Baza have been destroyed and disappeared respectively, and have made travel and living on the surface of Aanrah much more managable and consistant.
Beneath Aanrah's crust is the underground - a series of tunnels and cave systems that span from just below the surface down to the Underworld and Aanrah's core. These caves and tunnels are the home of a whole slew of countries and kingdoms that largely belong to the dwarves and dark elves, and contain some of the best roads for traveling that stretch back to Aanrah's early days when they were often the only way to reliably travel across tide-covered regions.
At Aanrah's center is the Underworld - a volcanic plane of reality that intersects through the deepest depths of Aanrah's underground. It is the home of the Nymauti demons and endbringers, and is known as being one of the two ways of leaving Aanrah itself.
Finally, there is the Faerealms - another plane of reality that intersects Aanrah. Eternally shrouded in twilight and surrounded by the Far Lands, the Faerealms are a land of exotic danger and home to the two fae races - the dangrous Nurí and the more friendly Siayi.

A Brief History of Aanrah
Aanrah's history is broken down into what are commonly known as "breaks" and eras - breaks being points between a change in Aanrah's moon situation, and eras being blocks of time that are considered to be ~500 years.
The Before Times
Aanrah's history starts before it was created. The First Gods were said to have gathered at a point in reality where power had gathered - a well upon which things could grow and feed, that attracted both fledgling divine and interlopers alike. Those who arrived came for different reasons - the fledglings sought a place to manifest their conciousness; the Nuríians sought a place in the Faerealms between the Far Lands where life grew and they could settle down; the Nymauti sought a place near a more stable point in space in which the Underworld was stable and they too could settle down; and the Outsiders simply sought power or even were just curious. Together, these groups discussed what could be done with this spring of power in reality, and eventually settled on the world's creation. Each with their own roles and plans, the First Gods created Aanrah and its three moons, as well as a few of the divineborn races.
The world is considered to have started in ernest and, thus become the first day and the first year when the ancient dragon god Bukrkry† split herself into five parts - Maholdymyndr, Deiahdrahl†, Droontastah†, Brástaath, and Jodiirook.

Break 1
The First Break is considered from the moment the world began to sometime shortly after Honuon's collapse. These very early years of Aanrah are not particularly well recorded, but it is known that great tide changes left terrestrial races with little land to build on and gave merfolk a much wider rule then they do in the modern era. Mortal races without inherent magic live in fear of those who did - in particular, dragons, Nuriians, Nymauti demons, and endbringers were considered very real and fatal threats.
This break came to an end when Yalidil Rana convinced Ineyale Phor† to aid her in attacking the first moon god, Morden Oeru†, in hopes of taking over his power and Honuon itself. While they successfully slew Morden, they did not account for how tightly bound he was to his moon, and for the 'birth' of Oreik from Morden's spilled blood. Though legends have conflicting verions of what happened, the reality is that with Morden dead, Honuon began to collapse. Yalidil saw Oreik as a easy scapegoat, and cursed him to be unable to speak the truth - though it's unlikely anyone would believe him anyways if he did say it. Oreik fled, and with the aid of exiled moon elves Iuin and Zhün, was able to open a brief portal between Honuon and Aanrah to allow some of Honuon's denizens to escape before their home fully collapsed. Oreik and the survivors would flee to the underground, Yalidil immediately pinning the blame upon him and Morden and the other gods, with no reason to distrust her, took to pursecuting the refugees.
Honuon's collapse would almost end the world due to the sheer size of the pieces of falling debris. A ascended dragon who would be later known as Léookobrah sacrificed himself to stop the largest piece of falling debris from likely wiping all life off of Aanrah's surface - an act that would revive him in some fashion as a Lesser God. Though the damage done by Honuon's collapse would still be terrible, life ultimately survived, and Aanrah continued.
The Siayi are thought to have appeared sometime during this break.

Break 2
The Second Break is considered the years between Honuon's collapse and Baza's disappearance. Early on in this break Oreik created the dark elves from the moon elven survivors of Honuon, and it is widely accepted that the ash elves were created by a unknown deity during this time as well. With one less moon pulling on Aanrah's tides, terrestrial races began to expand out more to areas once covered by high tides, and began to rebuild what was destroyed by the moon's falling debris.
Some of the greatest legends are said to have happened during this time, and many of the most well known Hero Deities were born and ascended in this period. Maholdymyndr would go on to slay Deiahdrahl during the 7th Era as his first step in consuming his brothers to return to the full power their mother had - a story captured by Bróbh Boíg, and the beginning of the Bard God's journey as the Witness.
The Second Break ultimately came to a end with the Progenitor of Mortal Magic - Szaifudrus. Born a mortal human king of a suffering country in modern-day Aranorin, Szaifudrus sought Emrul in an attempt to save his people. As legend would have it, Emrul gave Szaifudrus unparalleled knowledge of magic in return for his people's absolute worship of him, with the clause that Szaifudrus could not teach others what he'd been taught. Szaifudrus would break this promise, and as punishment Emrul would turn him into a monster and curse him to raze his home to the ground. Though Emrul desired to kill Szaifudrus and try to re-contain learned magic, Yalidil would intervene and use the entire moon of Baza to tie mortal magic to Szaifudrus, and make him impossible for Emrul to kill. Though the population of Aanrah is unaware that this is where Baza went, scholars do see the correlation of its disappearance with the story of Szaifudrus and the avalibility of learned magic.

Break 3
The Third Break is considered to be anything after Baza's vanishing and the 'start' of mortal magic. Despite Emrul's attempts, learned magic had spread too quickly for him to put back in the bag again. The access to magic began to allow mortals that once lived in fear of magically inclined races to begin to level the playing field - and, eventually, turn it around. Dragon slaying and monster hunting became a honored and popular job, and mortal races pushed back hard against the races that once had power over them in any form. Though Nymauti demons, endbringers, and Nuriians had easier places to hide, dragons were slowly and systematically pushed to near extinction.
In the 13th Era, Maholdymyndr managed to slay another one of his brothers, Droontastah. Though his goal was to still kill Brastaath and Jodiirook, the outlook for dragonkind began to look bleak. Instead of aiding, Maholdymyndr disappeared without a word, secretly burying himself into hiding within his volcanic throne of Gezralahdnat. Over the next few millenia, Brástaath and Jodiirook would also hide themselves away - but with much more reluctance, after their attempts to descilate the situation failed.

The Modern Era
In the modern era of Aanrah, mortals stand in a much stronger position then they ever have. Monsters, dragons, interlopers, and other once feared races now face pursecution in my surface countries, with dragons being considered largely extinct above ground. Magic has become a staple of daily life, now so readily avalible that anyone can learn it and simple spells are used by most. With learned magic's aid, techology has advanced as well - learned magic lends itself much better to enchanting and imparting upon things then inherent magic had. For most on Aanrah, life is better then it has ever been - most mortal races have a good quality of life, travel is easier then it ever has been, and so many things are now much more avalible and accessable to the wide population.
And with this, many dark plots brew. Some mortals wonder what need they might still have for gods when they wield the same power now themselves. Kingdoms fire their war engines to expand and conquer, now with new forms of horror to unleash on their enemies. Some races hide in fear of the majority population of Aanrah, now hunted out of irrational fear and a massively boosted sense of strength. There is a great amount of turmoil in the world, and it has been a very long time since the last break...

The Inhabitants of Aanrah
Aanrah is home to 22 distinct races and 2 informal races. Its races are considered to be split into three 'groups' - divinemade or born races, who were created by a god; worldmade or born races, who came to be with the world itself; and interlopers, who are races who originated on another plane of existence initially.

The Races
The interloper races are generally the oldest on Aanrah, given they existed before Aanrah itself existed, and includes the outsiders, Nuriians, Nymauti demons, endbringers, and Siayi. These race's inherent abilities make them all very dangerous, and as a result most are highly feared and prosecuted in the modern times. These races are also uncommon sights on Aanrah's surface, even before their pursecution, preferring to stick close to their home planes.
The divinemade races are otherwise the oldest races truly native to Aanrah, and include the ash elves, dark elves, dragons, high elves, merfolk, moon elves, sun elves, and wood elves. These races are known to have inherent magic of different degrees, though most (with the exception of dragons) do not have it to the same degree that many interloper species might. They make up a solid portion of Aanrah's world population, and as a whole do not have the same antagonistic relationship with each other and the worldborn races as interlopers do.
The worldborn races all came to be with the start of the world, though it is not understood how or why. These include the birdfolk, dwarves, gnomes, goblins, humans, monsters, naga, orcs, ratfolk, and sphinx. With the exception of sphinx and some monsters, worldborn races have no inherent magic, and tended to be at the bottom of the racial pecking order until the Third Break. They make up a majority of Aanrah's world population.

The Cultures \& Languages
Each race has a 'default' culture name and matching language. While any race can come from any culture or speak any language, these culture/language pairs are considered the 'baseline' for a member of each race born in the locale considered their race's 'original' home (i.e. where their populations are often most dense and may have initially started out in). While these are often grouped together as race/culture/language, they are in no means exclusive to each other, or mean that a member of any race by default knows that language or has the views of that culture.

The Divine
The divine on Aanrah are ever-present and very, very involved. They are major players in the world and are very frequently very active parts of mortal live in some form. On Aanrah, the divine tend to remain manifested in some form, albeit it is most frequently on a layer of reality that is just slightly offset from the one mortals live on. Meeting the divine is not as uncommon as one might expect, for the better or worse. And though the gods overall try to hold each other accountable and to discourage directly interfering with mortal matters, they are known to still step in and do things themselves instead of relying on worshippers.

Important Figures
While Aanrah does not have a main character or single story, it does have a variety of important characters and individuals. Highlighted and focused on characters will vary in importance to the overall world history and story.